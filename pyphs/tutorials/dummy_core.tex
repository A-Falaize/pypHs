%
\documentclass[11pt, oneside]{article}      % use 'amsart' instead of 'article' for AMSLaTeX format
\usepackage{geometry}                       % See geometry.pdf to learn the layout options. There are lots.
\geometry{letterpaper}                      % ... or a4paper or a5paper or ...
%\geometry{landscape}                       % Activate for for rotated page geometry
\usepackage[parfill]{parskip}               % Activate to begin paragraphs with an empty line rather than an indent
\usepackage{graphicx}                       % Use pdf, png, jpg, or eps with pdflatex; use eps in DVI mode
                                        % TeX will automatically convert eps --> pdf in pdflatex
\usepackage{amssymb}
%\date{\today}                              % Activate to display a given date or no date
\title{a Dummy PHSCore}
%
\usepackage{authblk}
\usepackage{hyperref}
%\renewcommand\Authands{ and }
%
%
\author[1]{The \textsc{PyPHS}\footnote{\url{https://afalaize.github.io/pyphs/}} development team}
%
\affil[1]{Project-team S3AM\footnote{\url{https://www.ircam.fr/recherche/equipes-recherche/systemes-et-signaux-sonores-audioacoustique-instruments-s3am/}}\\STMS, IRCAM-CNRS-UPMC (UMR 9912)\\1 Place Igor-Stravinsky, 75004 Paris, France}
%
\begin{document}
%
\maketitle
%
%
\section{System dimensions}
%
$\dim(\mathbf{x})=$ $ n_\mathbf{x} = 2 ; $ 
%
\\
%
$\dim(\mathbf{w})=$ $ n_\mathbf{w} = 0 ; $ 
%
\\
%
$\dim(\mathbf{y})=$ $ n_\mathbf{y} = 1 ; $ 
%
\\
%
$\dim(\mathbf{p})=$ $ n_\mathbf{p} = 0 ; $ 
%
\\
%
%
\section{System variables}
%
State variable $ \mathbf{x} = \left(\begin{array}{c}x_{\mathrm{L}}\\x_{\mathrm{C}}\end{array}\right) ; $ 
%
\\
%
Input $ \mathbf{u} = \left(\begin{array}{c}u_{\mathrm{out}}\end{array}\right) ; $ 
%
\\
%
Output $ \mathbf{y} = \left(\begin{array}{c}y_{\mathrm{out}}\end{array}\right) ; $ 
%
\\
%
%
\section{Constitutive relations}
%
Hamiltonian $ \mathtt{H}(\mathbf{x}) = \frac{x_{\mathrm{L}}^{2}}{2 \cdot L} + \frac{x_{\mathrm{C}}^{2}}{C} \cdot \left(\frac{Cnl}{4} \cdot x_{\mathrm{C}}^{2} + 0.5\right) ; $ 
%
\\
%
Hamiltonian gradient $ \nabla \mathtt{H}(\mathbf{x}) = \left(\begin{array}{c}\frac{x_{\mathrm{L}}}{L}\\\frac{1.0}{C} \cdot x_{\mathrm{C}} \cdot \left(Cnl \cdot x_{\mathrm{C}}^{2} + 1\right)\end{array}\right) ; $ 
%
\\
%
%
\section{System parameters}
%
%
\subsection{Constant}
%
\begin{center}
%
\begin{tabular}{ll}
%
\hline
parameter & value (SI)
\\ \hline
Cnl :& 100000000.0
\\
Rnl :& 100.0
\\
C :& 5.06605918212e-06
\\
L :& 0.5
\\
\hline
\end{tabular}
%
\end{center}
%
\section{System structure}
%

%
$ \mathbf{M} = \left(\begin{array}{ccc}- 1.0 \cdot Rnl \cdot \left(x_{\mathrm{L}}^{2} + 1\right) & -1.0 & -1.0\\1.0 & gxC & 0\\1.0 & 0 & 0\end{array}\right) ; $ 
%
\\
%
$ \mathbf{M_{xx}} = \left(\begin{array}{cc}- 1.0 \cdot Rnl \cdot \left(x_{\mathrm{L}}^{2} + 1\right) & -1.0\\1.0 & gxC\end{array}\right) ; $ 
%
\\
%
$ \mathbf{M_{xy}} = \left(\begin{array}{c}-1.0\\0\end{array}\right) ; $ 
%
\\
%
$ \mathbf{M_{yx}} = \left(\begin{array}{cc}1.0 & 0\end{array}\right) ; $ 
%
\\
%
$ \mathbf{M_{yy}} = \left(\begin{array}{c}0\end{array}\right) ; $ 
%
\\
%

%
$ \mathbf{J} = \left(\begin{array}{ccc}0 & -1.0 & -1.0\\1.0 & 0 & 0\\1.0 & 0 & 0\end{array}\right) ; $ 
%
\\
%
$ \mathbf{J_{xx}} = \left(\begin{array}{cc}0 & -1.0\\1.0 & 0\end{array}\right) ; $ 
%
\\
%
$ \mathbf{J_{xy}} = \left(\begin{array}{c}-1.0\\0\end{array}\right) ; $ 
%
\\
%
$ \mathbf{J_{yy}} = \left(\begin{array}{c}0\end{array}\right) ; $ 
%
\\
%

%
$ \mathbf{R} = \left(\begin{array}{ccc}1.0 \cdot Rnl \cdot \left(x_{\mathrm{L}}^{2} + 1\right) & 0 & 0\\0 & - 1.0 \cdot gxC & 0\\0 & 0 & 0\end{array}\right) ; $ 
%
\\
%
$ \mathbf{R_{xx}} = \left(\begin{array}{cc}1.0 \cdot Rnl \cdot \left(x_{\mathrm{L}}^{2} + 1\right) & 0\\0 & - 1.0 \cdot gxC\end{array}\right) ; $ 
%
\\
%
$ \mathbf{R_{xy}} = \left(\begin{array}{c}0\\0\end{array}\right) ; $ 
%
\\
%
$ \mathbf{R_{yy}} = \left(\begin{array}{c}0\end{array}\right) ; $ 
%
\\
%
\end{document}