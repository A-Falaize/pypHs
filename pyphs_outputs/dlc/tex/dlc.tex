%
\documentclass[11pt, oneside]{article}      % use 'amsart' instead of 'article' for AMSLaTeX format
\usepackage{geometry}                       % See geometry.pdf to learn the layout options. There are lots.
\geometry{letterpaper}                      % ... or a4paper or a5paper or ...
%\geometry{landscape}                       % Activate for for rotated page geometry
\usepackage[parfill]{parskip}               % Activate to begin paragraphs with an empty line rather than an indent
\usepackage{graphicx}                       % Use pdf, png, jpg, or eps with pdflatex; use eps in DVI mode
                                        % TeX will automatically convert eps --> pdf in pdflatex
\usepackage{amssymb}
%\date{\today}                              % Activate to display a given date or no date
\title{DLC}
%
\usepackage{authblk}
\usepackage{hyperref}
%\renewcommand\Authands{ and }
%
%
\author[1]{The \textsc{PyPHS}\footnote{\url{https://github.com/afalaize/pyphs}} development team}
%
\affil[1]{Project-team S3\footnote{\url{http://s3.ircam.fr}}\\STMS, IRCAM-CNRS-UPMC (UMR 9912)\\1 Place Igor-Stravinsky, 75004 Paris, France}
%
\begin{document}
%
\maketitle
%
%
\section{System netlist}
%
%
\begin{center}
%
\texttt{
\begin{tabular}{llllll}
\hline
line & label & dictionary.component & nodes & parameters \\ \hline
%
$\ell_1$ & in & electronics.source & ('\#', 'n1') & $\left\{ 
%
\begin{tabular}{ll}
%
type & voltage
\\
\end{tabular}\right.$
 \\
$\ell_2$ & D & electronics.diode & ('n1', 'n2') & $\left\{ 
%
\begin{tabular}{ll}
%
v0 & ('v0', 0.026)
\\
mu & ('mu', 1.7)
\\
Is & ('Is', 2e-09)
\\
R & ('Rd', 0.5)
\\
\end{tabular}\right.$
 \\
$\ell_3$ & L & electronics.inductor & ('n2', 'n3') & $\left\{ 
%
\begin{tabular}{ll}
%
L & ('L', 0.05)
\\
\end{tabular}\right.$
 \\
$\ell_4$ & C & electronics.capacitor & ('n3', '\#') & $\left\{ 
%
\begin{tabular}{ll}
%
C & ('C', 2e-06)
\\
\end{tabular}\right.$
 \\
\hline
\end{tabular}
%
}
%
\end{center}
%

%
%
\section{System dimensions}
%
$\dim(\mathbf{x})=$ $ n_\mathbf{x} = 2 ; $ 
%
\\
%
$\dim(\mathbf{w})=$ $ n_\mathbf{w} = 3 ; $ 
%
\\
%
$\dim(\mathbf{y})=$ $ n_\mathbf{y} = 1 ; $ 
%
\\
%
$\dim(\mathbf{p})=$ $ n_\mathbf{p} = 0 ; $ 
%
\\
%
%
\section{System variables}
%
State variable $ \mathbf{x} = \left(\begin{array}{c}x_{\mathrm{L}}\\x_{\mathrm{C}}\end{array}\right) ; $ 
%
\\
%
Dissipation variable $ \mathbf{w} = \left(\begin{array}{c}w_{\mathrm{D}}\\w_{\mathrm{D_R}}\\w_{\mathrm{D_gmin}}\end{array}\right) ; $ 
%
\\
%
Input $ \mathbf{u} = \left(\begin{array}{c}u_{\mathrm{in}}\end{array}\right) ; $ 
%
\\
%
Output $ \mathbf{y} = \left(\begin{array}{c}y_{\mathrm{in}}\end{array}\right) ; $ 
%
\\
%
%
\section{Constitutive relations}
%
Hamiltonian $ \mathtt{H}(\mathbf{x}) = \frac{0.5 x_{\mathrm{L}}^{2}}{L} + \frac{0.5 x_{\mathrm{C}}^{2}}{C} ; $ 
%
\\
%
Hamiltonian gradient $ \nabla \mathtt{H}(\mathbf{x}) = \left(\begin{array}{c}\frac{1.0 x_{\mathrm{L}}}{L}\\\frac{1.0 x_{\mathrm{C}}}{C}\end{array}\right) ; $ 
%
\\
%
Dissipation function $ \mathbf{z}(\mathbf{w}) = \left(\begin{array}{c}\mu v_{0} \log{\left (1 + \frac{w_{\mathrm{D}}}{Is} \right )}\\Rd w_{\mathrm{D_R}}\\\frac{w_{\mathrm{D_gmin}}}{gmin}\end{array}\right) ; $ 
%
\\
%
Jacobian of dissipation function $ \mathcal{J}_{\mathbf{z}}(\mathbf{w}) = \left(\begin{array}{ccc}\frac{\mu v_{0}}{Is \left(1 + \frac{w_{\mathrm{D}}}{Is}\right)} & 0 & 0\\0 & Rd & 0\\0 & 0 & \frac{1}{gmin}\end{array}\right) ; $ 
%
\\
%
%
\section{System parameters}
%
%
\subsection{Constant}
%
\begin{center}
%
\begin{tabular}{ll}
%
\hline
parameter & value (SI)
\\ \hline
C :& 2e-06
\\
v0 :& 0.026
\\
L :& 0.05
\\
mu :& 1.7
\\
gmin :& 1e-12
\\
Is :& 2e-09
\\
Rd :& 0.5
\\
\hline
\end{tabular}
%
\end{center}
%
\section{System structure}
%

%
$ \mathbf{M} = \left(\begin{array}{cccccc}0 & -1.0 & 0 & -1.0 & -1.0 & -1.0\\1.0 & 0 & 0 & 0 & 0 & 0\\0 & 0 & 0 & 0 & 1.0 & 0\\1.0 & 0 & 0 & 0 & 0 & 0\\1.0 & 0 & -1.0 & 0 & 0 & 0\\1.0 & 0 & 0 & 0 & 0 & 0\end{array}\right) ; $ 
%
\\
%
$ \mathbf{M_{xx}} = \left(\begin{array}{cc}0 & -1.0\\1.0 & 0\end{array}\right) ; $ 
%
\\
%
$ \mathbf{M_{xw}} = \left(\begin{array}{ccc}0 & -1.0 & -1.0\\0 & 0 & 0\end{array}\right) ; $ 
%
\\
%
$ \mathbf{M_{xy}} = \left(\begin{array}{c}-1.0\\0\end{array}\right) ; $ 
%
\\
%
$ \mathbf{M_{wx}} = \left(\begin{array}{cc}0 & 0\\1.0 & 0\\1.0 & 0\end{array}\right) ; $ 
%
\\
%
$ \mathbf{M_{ww}} = \left(\begin{array}{ccc}0 & 0 & 1.0\\0 & 0 & 0\\-1.0 & 0 & 0\end{array}\right) ; $ 
%
\\
%
$ \mathbf{M_{wy}} = \left(\begin{array}{c}0\\0\\0\end{array}\right) ; $ 
%
\\
%
$ \mathbf{M_{yx}} = \left(\begin{array}{cc}1.0 & 0\end{array}\right) ; $ 
%
\\
%
$ \mathbf{M_{yw}} = \left(\begin{array}{ccc}0 & 0 & 0\end{array}\right) ; $ 
%
\\
%
$ \mathbf{M_{yy}} = \left(\begin{array}{c}0\end{array}\right) ; $ 
%
\\
%

%
$ \mathbf{J} = \left(\begin{array}{cccccc}0 & -1.0 & 0 & -1.0 & -1.0 & -1.0\\1.0 & 0 & 0 & 0 & 0 & 0\\0 & 0 & 0 & 0 & 1.0 & 0\\1.0 & 0 & 0 & 0 & 0 & 0\\1.0 & 0 & -1.0 & 0 & 0 & 0\\1.0 & 0 & 0 & 0 & 0 & 0\end{array}\right) ; $ 
%
\\
%
$ \mathbf{J_{xx}} = \left(\begin{array}{cc}0 & -1.0\\1.0 & 0\end{array}\right) ; $ 
%
\\
%
$ \mathbf{J_{xw}} = \left(\begin{array}{ccc}0 & -1.0 & -1.0\\0 & 0 & 0\end{array}\right) ; $ 
%
\\
%
$ \mathbf{J_{xy}} = \left(\begin{array}{c}-1.0\\0\end{array}\right) ; $ 
%
\\
%
$ \mathbf{J_{ww}} = \left(\begin{array}{ccc}0 & 0 & 1.0\\0 & 0 & 0\\-1.0 & 0 & 0\end{array}\right) ; $ 
%
\\
%
$ \mathbf{J_{wy}} = \left(\begin{array}{c}0\\0\\0\end{array}\right) ; $ 
%
\\
%
$ \mathbf{J_{yy}} = \left(\begin{array}{c}0\end{array}\right) ; $ 
%
\\
%

%
$ \mathbf{R} = \left(\begin{array}{cccccc}0 & 0 & 0 & 0 & 0 & 0\\0 & 0 & 0 & 0 & 0 & 0\\0 & 0 & 0 & 0 & 0 & 0\\0 & 0 & 0 & 0 & 0 & 0\\0 & 0 & 0 & 0 & 0 & 0\\0 & 0 & 0 & 0 & 0 & 0\end{array}\right) ; $ 
%
\\
%
$ \mathbf{R_{xx}} = \left(\begin{array}{cc}0 & 0\\0 & 0\end{array}\right) ; $ 
%
\\
%
$ \mathbf{R_{xw}} = \left(\begin{array}{ccc}0 & 0 & 0\\0 & 0 & 0\end{array}\right) ; $ 
%
\\
%
$ \mathbf{R_{xy}} = \left(\begin{array}{c}0\\0\end{array}\right) ; $ 
%
\\
%
$ \mathbf{R_{ww}} = \left(\begin{array}{ccc}0 & 0 & 0\\0 & 0 & 0\\0 & 0 & 0\end{array}\right) ; $ 
%
\\
%
$ \mathbf{R_{wy}} = \left(\begin{array}{c}0\\0\\0\end{array}\right) ; $ 
%
\\
%
$ \mathbf{R_{yy}} = \left(\begin{array}{c}0\end{array}\right) ; $ 
%
\\
%
\end{document}