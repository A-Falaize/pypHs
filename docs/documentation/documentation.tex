\documentclass[10pt,a4paper]{article}
\usepackage[utf8]{inputenc}
\usepackage[francais]{babel}
\usepackage[T1]{fontenc}
\usepackage{amsmath}
\usepackage{amsfonts}
\usepackage{amssymb}
\author{Antoine Falaize}
\newcommand{\version}{0.1.9b2}
\title{\textsc{PyPHS Documentation}\\ Version \version}
% Default fixed font does not support bold face
\DeclareFixedFont{\ttb}{T1}{txtt}{bx}{n}{9} % for bold
\DeclareFixedFont{\ttm}{T1}{txtt}{m}{n}{9}  % for normal

% Custom colors
\usepackage{color}
\definecolor{deepblue}{rgb}{0,0,0.5}
\definecolor{deepred}{rgb}{0.6,0,0}
\definecolor{deepgreen}{rgb}{0,0.5,0}

\usepackage{listings}

% Python style for highlighting
\newcommand\pythonstyle{\lstset{
language=Python,
basicstyle=\ttm,
otherkeywords={self},             % Add keywords here
keywordstyle=\ttb\color{deepblue},
emph={MyClass,__init__},          % Custom highlighting
emphstyle=\ttb\color{deepred},    % Custom highlighting style
stringstyle=\color{deepgreen},
frame=tb,                         % Any extra options here
showstringspaces=false            % 
}}


% Python environment
\lstnewenvironment{python}[1][]
{
\pythonstyle
\lstset{#1}
}
{}

% Python for external files
\newcommand\pythonexternal[2][]{{
\pythonstyle
\lstinputlisting[#1]{#2}}}

% Python for inline
\newcommand\pythoninline[1]{{\pythonstyle\lstinline!#1!}}

%%%%%%% USAGE %%%%%%%
%\begin{document}
%
%\section{``In-text'' listing highlighting}
%
%\begin{python}
%class MyClass(Yourclass):
%    def __init__(self, my, yours):
%        bla = '5 1 2 3 4'
%        print bla
%\end{python}
%
%\section{External listing highlighting}
%
%\pythonexternal{demo.py}
%
%\section{Inline highlighting}
%
%Definition \pythoninline{class MyClass} means \dots
%
%\end{document}
\begin{document}
\renewcommand{\vec}[1]{\mathbf{#1}}
\newcommand{\mat}[1]{\mathbf{#1}}
\newcommand{\lab}[1]{\mathtt{#1}}
\newcommand{\op}[1]{\mathrm{#1}}

\newcommand{\Dpar}[2]{\frac{\partial #1}{\partial #2}}
\newcommand{\D}[2]{\frac{\op{d} #1}{\op{d} #2}}
\newcommand{\T}{\intercal}

\renewcommand{\d}{\op{d}}
\newcommand{\vx}{{\vec{x}}}
\newcommand{\vw}{{\vec{w}}}
\newcommand{\vz}{{\vec{z}}}
\newcommand{\vu}{{\vec{u}}}
\newcommand{\vy}{{\vec{y}}}
\newcommand{\vp}{{\vec{p}}}
\newcommand{\vg}{{\vec{g}}}
\newcommand{\vc}{{\vec{c}}}

\newcommand{\vvl}{{\vec{v}_{\lab l}}}
\newcommand{\vvnl}{{\vec{v}_{\lab{nl}}}}

\newcommand{\vfl}{{\vec{f}_{\lab l}}}
\newcommand{\vfnl}{{\vec{f}_{\lab{nl}}}}

\renewcommand{\H}{{\op{H}}}
\newcommand{\Z}[1]{\mat{Z}_{\lab{#1}}}
\newcommand{\J}[1]{\mat{J}_{\lab{#1}}}
\newcommand{\R}[1]{\mat{R}_{\lab{#1}}}
\newcommand{\M}[1]{\mat{M}_{\lab{#1}}}
\newcommand{\N}[1]{\mat{N}_{\lab{#1}}}
\newcommand{\dxH}[1]{\nabla\H_{\lab{#1}}}
\newcommand{\dxHd}[1]{\overline{\nabla}\H_{\lab{#1}}}
\newcommand{\Id}{\mat{I_d}}

\newcommand{\diag}{\operatorname{diag}}

\newcommand{\Q}{\mat{Q}}

\newcommand{\vxl}{\vx_{\lab l}}
\newcommand{\vxnl}{\vx_{\lab{nl}}}
\newcommand{\Hl}{\H_{\lab l}}
\newcommand{\Hnl}{\H_{\lab{nl}}}
\newcommand{\nx}{{n_{\lab{x}}}}
\newcommand{\nxl}{{n_{\lab{xl}}}}
\newcommand{\nxnl}{{n_{\lab{xnl}}}}

\newcommand{\vwl}{\vw_{\lab l}}
\newcommand{\vwnl}{\vw_{\lab{nl}}}
\newcommand{\vzl}{\vz_{\lab l}}
\newcommand{\vznl}{\vz_{\lab{nl}}}
\newcommand{\nw}{{n_{\lab{w}}}}
\newcommand{\nwl}{{n_{\lab{wl}}}}
\newcommand{\nwnl}{{n_{\lab{wnl}}}}

\newcommand{\e}{\mathfrak{e}}
\newcommand{\f}{\mathfrak{f}}
\newcommand{\ve}{\boldsymbol{\e}}
\newcommand{\vf}{\boldsymbol{\f}}

\newcommand{\ny}{n_{\lab{y}}}
\newcommand{\nc}{n_{\lab{y}}}
\renewcommand{\ng}{n_{\lab{g}}}
\newcommand{\np}{n_{\lab{y}}}

\newcommand{\ones}{ \mathbbold{1}}
\newcommand{\zeros}{ \mathbbold{0}}

\newcommand{\RR}{\mathbb{R}}
\newcommand{\CC}{\mathbb{C}}
\newcommand{\NN}{{\mathbb{N}}}
\newcommand{\ZZ}{{\mathbb{Z}}}


%%%%%%
%Fractional Calculus Commandes 
\newcommand{\bmu}{\boldsymbol{\mu}}
\newcommand{\Icx}{{\dot\iota}}
\newcommand{\HH}{{\mathbb{H}}}
\newcommand{\cC}{{\mathcal{C}}}
\newcommand{\EXP}[1]{{\mathrm{e}^{#1}}}
\newcommand{\myvector}[1]{{\mathbf{#1}}}%{{\underline{#1}}}
\newcommand{\Ev}[0]{\myvector{E}}
\newcommand{\Hv}[0]{\myvector{H}}
\newcommand{\Mv}[0]{\myvector{M}}
\newcommand{\Vv}[0]{\myvector{V}}
\newcommand{\Wv}[0]{\myvector{W}}
\newcommand{\herm}[1]{{\overline{#1}^\T}}
\newcommand{\npoles}{{n_\xi}}
\newcommand{\nfreqs}{{n_\omega}}
\newcommand{\fracint}{{\mathcal{I}}}
\newcommand{\fracdiff}{{\mathcal{D}}}

%
\maketitle
%
\section{Introduction}
%
The python package \pythoninline{pyphs} is dedicated to the treatment of \emph{passive multiphysical systems} in the \emph{Port-Hamiltonian Systems} (PHS) formalism. This formalism structures physical systems into 
%
\begin{itemize}
\item energy conserving parts, 
\item power dissipating parts and 
\item source parts.
\end{itemize}
%

This guarantees a \emph{power balance} is fulfilled, including for numerical \emph{simulations} based on an adapted \emph{numerical method}.   
%

\begin{enumerate}
\item Systems are described by \emph{directed multi-graphs} (\pythoninline{networkx.MultiDiGraph}).
\item The time-continuous port-Hamiltonian structure is build from an \emph{automated graph analysis}.
\item The discrete-time port-Hamiltonian structure is derived from a \emph{structure preserving numerical method}.
\item \LaTeX description code and \textsc{C++} simulation code are automatically generated.
\end{enumerate}
\subsection{Installation}
%
\begin{center}
\textbf{Notice only python 2.7 is supported.}
\end{center}
%
It is recommanded to install \pythoninline{pyphs} using \textsc{PyPI} (the \textsc{Python Package Index}). In terminal:
%
\begin{verbatim}
pip install pyphs
\end{verbatim}
%
\textbf{Mac OSX only:} An installation for \emph{Anaconda} users is also available. In terminal:
%
\begin{verbatim}
conda install -c afalaize pyphs
\end{verbatim}
%
\subsection{The PHS formalism}
%
Below is a recall of the Port-Hamiltonian Systems (PHS) formalism.
%
For details, the reader is referred to the \textit{e.g.} the acaemic reference \cite{falaize2016apassive}.
%
%
We consider systems that can be described by the following time-continuous non-linear state-space representation:
%
%
\begin{equation}
\underbrace{\left(
\begin{array}{c}
\D{\vx}{t}\\
\vw\\
\vy
\end{array}
\right)}_{\vec b}
=
\underbrace{\left(\begin{array}{lll}
\M{xx}&\M{xw} & \M{xy}\\ 
\M{wx}&\M{ww} & \M{wy}\\ 
\M{yx}&\M{yw} & \M{yy}
\end{array}\right)}_{\M{}}
\cdot
\underbrace{\left(
\begin{array}{c}
\dxH{}(\vx)\\
\vz(\vw)\\
\vu
\end{array}
\right)
}_{\vec a}
\end{equation}
%
where 
\begin{equation}
\M{}
=
\underbrace{\left(\begin{array}{lll}
\J{xx}&\J{xw} & \J{xy}\\ 
\J{wx}&\J{ww} & \J{wy}\\ 
\J{yx}&\J{yw} & \J{yy}
\end{array}\right)}_{\J{}}
-
\underbrace{\left(\begin{array}{lll}
\R{xx}&\R{xw} & \R{xy}\\ 
\R{wx}&\R{ww} & \R{wy}\\ 
\R{yx}&\R{yw} & \R{yy}
\end{array}\right)}_{\R{}}
\end{equation}
%
and
%
\begin{itemize}
\item $\J{}:\vx \mapsto \J{}(\vx)$ is a skew-symmetric matrix: $$\J{\alpha\beta}=-\J{\beta\alpha}^\T \mbox{~~for~~} (\alpha, \beta)\in\{\lab x, \lab w, \lab y\}^2,$$
\item $\R{}:\vx \mapsto \R{}(\vx)\succeq 0$ is a positive definite matrix,
\item $\vx: t \mapsto \vx(t)\in\RR^{\nx}$ is the \emph{state vector},
\item $\H:\vx \mapsto \H(\vx)\in \RR_+$ is a \emph{storage function} (convex and positive-definite scalar function with $\H(0)=0$),
\item $\nabla\H:\vx\mapsto\nabla \H(\vx)\in\RR^{\nx}$ denote the gradient of the storage function with the \emph{storage power} $$\op P_\vx = \D{\vx}{t}\cdot \nabla\H(\vx),$$
\item $\vw: t \mapsto \vw(t)\in\RR^{\nw}$ is the \emph{dissipation vector variable},
\item $\vz:\vw \mapsto \vz(\vw)\in \RR^{\nw}$ is a \emph{dissipation function} (with positive definite jacobian matrix and $\vz(0)=0$) for the \emph{dissipated power} $$\op P_\vw = \vw\cdot \vz(\vw) + \vec a\cdot\R{}\cdot\vec a,$$
\item $\vu: t \mapsto \vu(t)\in\RR^{\ny}$ is the \emph{input vector},
\item $\vy: t \mapsto \vy(t)\in\RR^{\ny}$ is the \emph{output vector},
\item  \textbf{the power received \emph{by} the sources \emph{from} the system is} $$\op P=\vu\cdot\vy.$$
\end{itemize}
%

The state is split according to $\vx=(\vxl^\T, \,\vxnl^\T)^\T $ with 
%

\begin{description}
%
\item[$\vxl = (x_1,\cdots,\,x_\nxl)^\T$] the states associated with the quadratic components of the storage function $\Hl(\vxl)=\frac{\vxl\cdot\Q\cdot\vxl}{2}$
%
\item[$\vxnl=(x_{\nxl+1},\cdots,\,x_\nx)^\T$] the states associated with the non-quadratic components of the storage function with $\nx = \nxl+\nxnl$ and $$\H(\vx) = \Hl(\vxl) + \Hnl(\vxnl)$$.\\
%
\end{description}
%

%
The set of dissipative variables is split according to $\vw = (\vxl^\T, \, \vwnl^\T)^\T$ with 
%
\begin{description}
%
\item[$\vwl = (w_1,\cdots,\,w_\nwl)^\T$] the variables associated with the linear components of the dissipative relation $\vzl(\vwl)=\mat{Z}_{\lab l}\,\vwl$
%
\item[$\vwnl=(w_{\nwl+1},\cdots,\,w_\nw)^\T$] the variables associated with the nonlinear components of the dissipative relation $\vznl:\vwnl\mapsto\vznl(\vwnl)\in\RR^{\nwnl}$ with $\nw = \nwl+\nwnl$ and $$ \vz(\vw) = \left(\begin{array}{c}
\mat{Z}_{\lab l}\,\vwl\\
\vznl(\vwnl)
\end{array} \right).$$
%
\end{description}
%

Accordingly, the structure matrices are split as
%
\begin{equation}
\underbrace{\left(\begin{array}{c}
\D{\vxl}{t}\\
\D{\vxnl}{t}\\ \hline
\vwl\\
\vwnl\\ \hline
\vy
\end{array}\right)}_{\vec b}
 = \underbrace{\left(\begin{array}{ll|ll|l}
\M{xlxl}&\M{xlxnl}&\M{xlwl} & \M{xlwnl} & \M{xly}\\ 
\M{xnlxl}&\M{xnlxnl}&\M{xnlwl} &  \M{xnlwnl} & \M{xnly}\\ \hline
\M{wlxl}& \M{wlxnl}& \M{wlwl} & \M{wlwnl} & \M{wly}\\
\M{wnlwl}& \M{wnlxnl}& \M{wnlwl} & \M{wnlwnl} & \M{wnly}\\ \hline
\M{yxl} &\M{yxnl}&\M{ywl} & \M{ywnl} &\M{yy}
\end{array}\right)}_{\M{}}
\cdot
\underbrace{\left(\begin{array}{c}
\Q\cdot\vx\\
\nabla\Hnl(\vxnl)\\ \hline
\mat{Z}_{\lab l}\cdot\vwl\\
\vznl(\vwnl)\\ \hline
\vu
\end{array}\right) }_{\vec a}
\end{equation}
%
\tableofcontents
%
\section{Structure of the \texttt{pyphs.PortHamiltonianObject}}
%
Below is a list of each module of practical use in the object  \pythoninline{pyphs.PortHamiltonianObject}, along with a short description. We consider the following instantiation:
%
\begin{python}
# import of (pre-installed) pyphs package:
import pyphs

# instantiate the PortHamiltonianObject:
phs = pyphs.PortHamiltonianObject(label='mylabel')
\end{python}
%
\subsection{The \texttt{symbs} module} Container for all the \textsc{SymPy} symbolic variables (\pythoninline{sympy.Symbol}). 
\begin{description}
%
\item[Attributes] are ordered \emph{list of symbols} associated with the system's vectors components.
%
\begin{description}
\item \pythoninline{phs.symbs.x}: state vector symbols $\vx \in \RR^{\nx}$,
\item \pythoninline{phs.symbs.w}: dissipative vector variable symbols $\vw\in\RR^{\nw}$,
\item \pythoninline{phs.symbs.u}: input vector symbols $\vu\in\RR^{\ny}$,
\item \pythoninline{phs.symbs.y}: output vector symbols $\vy \in \RR^{\ny}$,
\item \pythoninline{phs.symbs.cu}: input vector symbols for connectors $\mathbf{c_u}\in\RR^{\nc}$,
\item \pythoninline{phs.symbs.cy}: output vector symbols for connectors $\mathbf{c_y}\in\RR^{\nc}$,
\item \pythoninline{phs.symbs.p}: Time-varying parameters symbols $\vp\in \RR^{\np}$.
\end{description}
%
\item[Methods]:
%
\begin{description}
%
\item \pythoninline{phs.symbs.dx()}: Returns the symbols associated with the state differential $\d\vx$ formed by appending the prefix $d$ to each symbol in \texttt{x}.
%
\item \pythoninline{phs.symbs.args()}: Return the list of symbols associated with the vector of all arguments of the symbolic expressions (\texttt{expr} module).
%
\end{description}
%
\end{description}
%
\subsection{The \texttt{exprs} module} Container for all the \textsc{SymPy} symbolic expressions \pythoninline{sympy.Expr}associated with the system's functions.
\begin{description}
%
\item[Attributes:] For scalar function (e.g. the storage function $\H$), arguments of \pythoninline{phs.exprs} are \textsc{SymPy} expressions (\pythoninline{sympy.Expr}); for vector functions (e.g. the disipative function $\vz$), arguments are ordered lists of \textsc{SymPy} expressions; for matrix functions (e.g. the Jacobian matrix of disipative function $\vz$), arguments are \pythoninline{sympy.Matrix} objects. Notice the expressions arguments\footnote{Accessed through the \pythoninline{sympy.Expr.free\_symbols} (\textit{e.g.} \pythoninline{phs.exprs.H.free\_symbols} to recover the arguments of the Storage function $\H$).} must belong either to (i) the elements of \pythoninline{phs.symbs.args()}, or (ii) the keys of the dictionary \pythoninline{phs.symbs.subs}.
%
\begin{description}
\item \pythoninline{phs.exprs.H}: storage function $\H\in\RR$,
\item \pythoninline{phs.exprs.z}: dissipative function $\vz\in\RR^{\nw}$,
\item \pythoninline{phs.exprs.g}: input/output gains vector function $\vg \in \RR^{\ng}$,
\end{description}
%
The following expression are computed from the \texttt{exprs.build()} method (see below):
%
\begin{description}
\item \pythoninline{phs.exprs.dxH}: the continuous gradient vector of storage scalar function $\dxH{}(\vx)\in\RR^{\nx}$,
\item \pythoninline{phs.exprs.dxHd}: the discrete gradient vector of storage scalar function $\dxHd{}(\vx, \delta\vx)\in\RR^{\nx}$,
\item \pythoninline{phs.exprs.hessH}: the continuous hessian matrix of storage scalar function (computed as $\nabla\nabla\H(\vx)\in\RR^{\nx\times\nx}$),
\item \pythoninline{phs.exprs.jacz}: the continuous jacobian matrix of dissipative vector function $\nabla\vz(\vw)\in\RR^{\nw\times\nw}$.
\item \pythoninline{phs.exprs.y}: the expression of the continuous output vector function $\vy(\dxH{}, \vz, \vu)\in\RR^{\ny}$,
\item \pythoninline{phs.exprs.yd}: the expression of the discrete output vector function $\overline{\vy}(\dxHd{}, \vz, \vu)\in\RR^{\ny}$,
\end{description}
%
\item[Methods]:
%
\begin{description}
%
\item \pythoninline{phs.exprs.build()}: Build the following system functions as \textsc{SymPy} expressions and append them as attributes to the \pythoninline{phs.exprs} module: \pythoninline{phs.exprs.dxH}, \pythoninline{phs.exprs.dxHd}, \pythoninline{phs.exprs.hessH}, \pythoninline{phs.exprs.jacz}, \pythoninline{phs.exprs.y}, and \pythoninline{phs.exprs.yd}.
%
\item \pythoninline{phs.exprs.setexpr(name, expr)}: Add the \textsc{SymPy} expression \pythoninline{expr} to the \pythoninline{phs.exprs} module, with argument 
\pythoninline{name}, and add \pythoninline{name} to the set of \pythoninline{phs.exprs.\_names}.
%
\item \pythoninline{phs.exprs.freesymbols()}: Return a python set of all the free symbols (\pythoninline{sympy.Symbol}) that appear at least once in all expressions with names in \pythoninline{phs.exprs.\_names}.
\end{description}
%
\end{description}
%
\subsection{The \texttt{dims} module} 
Container for accessors to the system's dimensions. 
%
No attributes should be changed manually. 
%
To split the system into its linear and nonlinear part, use \pythoninline{phs.split_linear()} which organize the system vectors as
%
\begin{equation}
\vx = \left(
\begin{array}{c}
\vx_{\lab{l}} \\
\vx_{\lab{nl}}
\end{array}
\right), \quad \dim(\vx_{\lab{l}}) = 
\end{equation}
%
\begin{description}
%
\item[Attributes:] 
%
\begin{description}
\item \pythoninline{phs.dims.xl}: Number of state vector components associated with a quadratic storage function: $\H_{\lab{l}}(\vx_{\lab{l}}) = \vx_{\lab{l}}^\T\cdot\frac{\Q}{2}\cdot\vx_{\lab{l}}$, and \pythoninline{phs.dims.x()} is equal to \pythoninline{phs.dims.xl + phs.dims.xnl()}.
\item \pythoninline{phs.dims.wl}: Number of dissipative vector variable components associated with a linear dissipative function: $\vz_{\lab{l}}(\vw_{\lab{l}})=\Z{l}\cdot\vw_{\lab{l}}$, and \pythoninline{phs.dims.w()} is equal to \pythoninline{phs.dims.wl + phs.dims.wnl()}.
\end{description}
%
\item[Methods]:
%
\begin{description}
%
\item \pythoninline{phs.dims.x()}: Return the dimension of state vector \pythoninline{len(phs.symbs.x)}.
%
\item \pythoninline{phs.dims.xnl()}: Return the number of state vector components associated with a nonlinear storage function 

\pythoninline{len(phs.symbs.x)}.
%
\item[\texttt{setexpr(name, expr)}]: Add the \textsc{SymPy} expression \texttt{expr} to the \texttt{exprs} module, with argument 
\texttt{name}, and add \texttt{name} to the set of \texttt{exprs.\_names}.
%
\item[\texttt{freesymbols()}]: Retrun a python set of all the free symbols (\texttt{sympy.symbols}) that appear at least once in all expressions with names in \texttt{exprs.\_names}.
\end{description}
%
\end{description}
%
%
%
%
%
\section{Algorithms}
%
This section details the algorithms actually implemented for 
%
\begin{enumerate}
%
\item the graph analysis and\\
%
\item the different simulation methods
%
\end{enumerate}
%
\subsection{Graph analysis}
%
The graph analysis method that derives the port-Hamiltonian system's differential-algebraic equations from with a given netlist is detailed in the reference \cite{falaize2016apassive}.
%
The algorithm implemented in \textsc{PyPHS} is exactly that in \cite[algorithm 1]{falaize2016apassive}.
%
\subsection{Simulation methods}
%
The discrete gradient method is used in conjunction with the port-Hamiltonian structure to produce a passive-guaranteed numerical scheme (see \cite{falaize2016apassive} for details).
%
In the sequel, quantities are defined on the current time step $\vx\equiv\vx(t_k)$, with $k\in\mathbb{N}_{+}^{*}$.
%
\subsubsection{Split of the linear part from the nonlinear part}
%
The dicrete gradient for the quadratic part of the Hamiltonian is ${\small \nabla\Hl =\frac{1}{2}\,\mat{Q}\, (2\vxl + \delta \vxl)}$ and the discret linear subsystem is
 %
%
\begin{equation}
\begin{array}{rcl}
\mat{D}_{\lab l}^{-1} = \mat{iD}_{\lab l} &=& \left(\begin{array}{ll}
\frac{\Id}{\delta t}&0 \\ 
0 &\Id
\end{array}\right)-\left(\begin{array}{ll}
\M{xlxl}&\M{xlwl} \\ 
\M{wlxl}&\M{wlwl} 
\end{array}\right)\,\left(\begin{array}{ll}
\frac{1}{2}\,\mat{Q}&0 \\ 
0 &\mat{Z}_{\lab l}
\end{array}\right),\\
\underbrace{\left(\begin{array}{c}
\delta \vxl\\
\vwl
\end{array}\right) 
}_{\vec{v}_{\lab l}}
& = &
\underbrace{\mat{D}_{\lab l}\,
\underbrace{\left(\begin{array}{ll}
\M{xlxl}\\ 
\M{wlxl}
\end{array}\right)\,\Q}_{\overline{\N{lxl}}}}_{\N{lxl}}
\, 
\vxl
+
  \underbrace{\mat{D}_{\lab l}\,
\underbrace{\left(\begin{array}{ll}
\M{xlxnl}& \M{xlwnl} \\ 
 \M{wlxnl}& \M{wlwnl} 
\end{array}\right)}_{\overline{\N{lnl}}}}_{\N{lnl}}
\,
\underbrace{\left(\begin{array}{c}
\nabla\Hnl\\
\vznl
\end{array}\right) 
}_{\vec{f}_{\lab{nl}}}
+
  \underbrace{\mat{D}_{\lab l}\,
\underbrace{\left(\begin{array}{l}
 \M{xly}\\ 
 \M{wly}
\end{array}\right)}_{\overline{\N{ly}}}}_{\N{ly}}
\,
\vu
\end{array}
\end{equation}
%
and the nonlinear subsystem is 
%
\begin{equation}
\begin{array}{rcl}
\left(
\begin{array}{cc}
\frac{\Id}{\delta t} & 0 \\
0 & \Id
\end{array}
\right)\underbrace{\left(\begin{array}{c}
\delta \vxnl \\
\vwnl
\end{array}\right) 
}_{\vec v_{\lab{nl}}}
&=&
  \underbrace{\left(\begin{array}{ll}
\M{xnlxnl}& \M{xnlwnl} \\ 
 \M{wnlxnl}& \M{wnlwnl} 
\end{array}\right)}_{\overline{\N{nlnl}}}
\,
\vec{f}_{\lab {nl}}
+
  \underbrace{\left(\begin{array}{l}
 \M{xnly}\\ 
 \M{wnly}
\end{array}\right)}_{\overline{\N{nly}}}
\,
\vu \\
 && +
\underbrace{ \left(\begin{array}{ll}
\M{xnlxl}&\M{xnlwl} \\ 
\M{wnlxl}&\M{wnlwl} 
\end{array}\right)\,\left(\begin{array}{ll}
\frac{1}{2}\,\mat{Q}&0 \\ 
0 &\mat{Z}_{\lab l}
\end{array}\right)}_{\overline{\N{nll}}}
\, 
\vec v_{\lab{l}}
+
\underbrace{\left(\begin{array}{ll}
\M{xnlxl}\\ 
\M{wnlxl}
\end{array}\right)\,\Q}_{\overline{\N{nlxl}}}
\, 
\vxl
\end{array}
\end{equation}
\subsubsection{Presolve numerical nonlinear subsystem}
\begin{equation}
\begin{array}{rcl}
\left(
\begin{array}{cc}
\frac{\Id}{\delta t} & 0 \\
0 & \Id
\end{array}
\right)\,
\vec v_{\lab{nl}} &=& \underbrace{\left(\overline{\N{nlxl}} + \overline{\N{nll}}\,{\N{lxl}}\right)}_{\N{nlxl}}\,\vxl + \underbrace{\left(\overline{\N{nlnl}} + \overline{\N{nll}}\,{\N{lnl}}\right)}_{\N{nlnl}}\,\vec f_{\lab{nl}}\\
&& \underbrace{\left(\overline{\N{nly}}+ \overline{\N{nll}}\,{\N{ly}}\right)}_{\N{nly}}\,\vu
\end{array}
\end{equation}
%
\subsubsection{Algorithm}
\paragraph{Inputs}
\begin{equation}
\begin{array}{rcl}
\mat{iD}_{\lab l} &=& \left(\begin{array}{ll}
\frac{\Id}{\delta t}&0 \\ 
0 &\Id
\end{array}\right)-\left(\begin{array}{ll}
\M{xlxl}&\M{xlwl} \\ 
\M{wlxl}&\M{wlwl} 
\end{array}\right)\,\left(\begin{array}{ll}
\frac{1}{2}\,\mat{Q}&0 \\ 
0 &\mat{Z}_{\lab l}
\end{array}\right) 
\\
\overline{\N{lxl}}
&=&
\left(\begin{array}{ll}
\M{xlxl}\\ 
\M{wlxl}
\end{array}\right)\,\mat Q
\\
\overline{\N{lnl}}
&=&
\left(\begin{array}{ll}
\M{xlxnl} & \M{xlwnl}\\ 
\M{wlxnl} & \M{wlwnl}
\end{array}\right)
\\
\overline{\N{ly}}
&=&
\left(\begin{array}{ll}
\M{xly}\\ 
\M{wly}
\end{array}\right)
\\
\overline{\N{nlnl}}
&=&
\left(\begin{array}{ll}
\M{xnlxnl}& \M{xnlwnl} \\ 
 \M{wnlxnl}& \M{wnlwnl} 
\end{array}\right)
\\
\overline{\N{nll}}
&=&
\left(\begin{array}{ll}
\M{xnlxl}& \M{xnlwl} \\ 
 \M{wnlxl}& \M{wnlwl} 
\end{array}\right)
\,
\left(\begin{array}{ll}
\frac{1}{2}\Q	& 0\\ 
0& \mat Z_\lab l
\end{array}\right)
\\
\overline{\N{nlxl}}
&=&
\left(\begin{array}{ll}
\M{xnlxl}\\ 
 \M{wnlxl}
\end{array}\right)
\,
\Q
\\
\overline{\N{nly}}
&=&
\left(\begin{array}{ll}
\M{xnly}\\ 
 \M{wnly}
\end{array}\right)
\\
\mathcal{J}_{\vec f \lab {nl}}(\vec v_{\lab {nl}})
&=&
\left(\begin{array}{ll}
\mathcal{J}_{\nabla \Hnl}	& 0\\ 
0& \mathcal{J}_{\vznl}
\end{array}\right)
\\
\mat I_{\lab{nl}}
&=&
\left(
\begin{array}{cc}
\frac{\Id}{\delta t} & 0 \\
0 & \Id
\end{array}
\right)
\end{array}
\end{equation}
%
\paragraph{Process}
\begin{equation}
\begin{array}{rcl}
\mat D_{\lab l}&=& \mat{iD}_{\lab l}^{-1}\\
\N{lxl} &=& \mat D_{\lab l}\,\overline{\N{lxl}}\\
\N{lnl} &=& \mat D_{\lab l}\,\overline{\N{lnl}}\\
\N{ly} &=& \mat D_{\lab l}\,\overline{\N{ly}}\\
\N{nlxl} &=& \overline{\N{nlxl}} + \overline{\N{nll}}\,{\N{lxl}}\\
\N{nlnl} &=& \overline{\N{nlnl}} + \overline{\N{nll}}\,{\N{lnl}}\\
\N{nly} &=& \overline{\N{nly}} + \overline{\N{nll}}\,{\N{ly}}\\
\vec c &=& \N{nlxl}\,\vxl + \N{nly}\,\vu \\
\mbox{Iterate}  &:&\vec F_{\lab{nl}}(\vec v_{\lab{nl}})=\mat I_{\lab{nl}}\, \vec v_{\lab{nl}}-\N{nlnl}\,\vec f_{\lab{nl}} - \vec c \\
&&\mathcal{J}_{\vec F_{\lab{nl}}}(\vec v_{\lab{nl}}) 
=
\mat I_{\lab{nl}} - \N{nlnl}\,\mathcal{J}_{\vec f \lab {nl}}(\vec v_{\lab {nl}})\\
& & \vec v_{\lab{nl}} = \vec v_{\lab{nl}}-\mathcal{J}^{-1}_{\vec F_{\lab{nl}}}(\vec v_{\lab{nl}}) \,\vec F_{\lab{nl}}(\vec v_{\lab{nl}})\\
\vec v_{\lab{l}} &=& \N{lxl}\,\vxl +\N{lnl}\,\vec f_{\lab{nl}} + \N{ly}\,\vu\\
\vy & = &\M{yxl}\,\nabla \Hl + \M{yxnl}\,\nabla \Hnl\M{ywl}\,\mat Z_{\lab l}\,\vwl + \M{ywnl}\,\vznl + \M{yy}\,\vu\\
\vx & = &\vx+ \delta \vx
\end{array}
\end{equation}

\begin{eqnarray}
\vy & = &\M{yxl}\,\nabla \Hl + \M{yxnl}\,\nabla \Hnl\M{ywl}\,\mat Z_{\lab l}\,\vwl + \M{ywnl}\,\vznl + \M{yy}\,\vu\\
&=& \M{yxl}\,\nabla \Hl + \M{yxnl}\,\nabla \Hnl\M{ywl}\,\mat Z_{\lab l}\,\vwl + \M{ywnl}\,\vznl + \M{yy}\,\vu\\
\end{eqnarray}

\bibliographystyle{apalike}
\bibliography{documentation}
\end{document}